\section{System Design}

Tra introduced the novel idea of {\em vector time pairs}, which allows Tra to provide very strong guarantees such as {\em no false positives} and {\em no meta-data for deleted files} among others. Our main system design is completely modelled after Tra.  In order to ensure robustness and resilience to crashes, we create additional mechanisms as well. On a high level, TnT's design can be understood in terms of two main primitives: first, we implement a File Watcher based on inotify to keep checking for file system updates when Tra is online, secondly, in cases when Tra comes back online after a crash, it recovers its metadata from the disk and checks if any file system modifications have been made when TnT was offline.  We discuss these core aspects of our system below.

\subsection{File Watcher}
TnT structures synchronization meta-data in the form of a {\em tree}, which stores modification histories and synchronization histories for every file/directory in the file system. We use the \href{http://godoc.org/code.google.com/p/go.exp/inotify}{\tt inotify} linux package, that allows the system to recognize when a user makes changes to the file system (such as creates/edits/deletes of files/directories) in an interrupt based manner. This allows us to keep the meta-data up-to-date with the latest state of the file system.

One design challenge associated with this method is that during synchronization, the File Watcher must be able to distinguish between whether the directory is modified by the user or by a synchronization.  Since File Watcher is interrupt based, it must act based on the information in the interrupt.  The {\tt inotify} linux package returns the name of the file and an event mask describing the type of event that occured.  The name and event do not provide enough information.  In order to distinguish between a sync and user action, TnT first copies all of the changes to a {\em tmp} folder associated with its directory.  Then, the process copies all of its changes from {\em tmp} into the actual location. This allows the File Watcher to distinguish between a change made by the user as opposed to the synchronization protocol. The {\em tmp} folder also plays a crucial role in implementing the 2-phase commit mechanism, described in Subsection \ref{subsec:failure}

\subsection{Synchronization}
Along the lines of Tra, we consider only one-directional synchronization. Any two machines can synchronize their file systems independent of the rest of the machines. The vector time-pairs that we store in the meta-data allow us to check if there are any changes to be made in any particular file or inside the sub-tree of any directory. Thus, our synchronization protocol checks on the root if any change has happened in it's sub-tree. If yes, then we run the synchronization protocol recursively on all files/directories in the root. This ensures that the number of round trips of communication made is proportional to the size of the sub-tree that the two systems differ in. In particular, if only one file differs between the two synchronizing peers, then the number of round trips of communication required is equal to the depth of the file being sychronized.

\subsection{Failure Recovery} \label{subsec:failure}
We use a two-phase synchronization method to ensure that crashes do not affect the consistency of the system, even if they happen during sync. The synchronization scheme goes through the following steps:
\begin{enumerate}
\item Update your metadata and record the changes to be made due to the current sync. Move files from the remote machine to the temporary directory and mark files to be deleted.
\item Dump the metadata to the disk.
\item Apply the recorded changes in the log and move files from the temporary directory to the working directory. Delete the files to be deleted.
\item Dump the metadata to disk.
\end{enumerate}
This synchronization mechanism, on a crash, recovers the metadata from the disk and uses the metadata to update filesystem state if there are any changes that have not been applied. This ensures that crashes after/during any of these steps can be handled gracefully.\\
While recovering from a crash, TnT reads the filesystem state and figures out if any files have been modified since the last crash and updates their version vectors if that is the case. This feature is enabled by explicitly tracking the last modified time of each file and directory in the metadata.

\subsection{Meta-data costs}
We managed to remove the metadata costs for deleted files. We had to seek help from Russ Cox to understand why his scheme works for avoiding meta data costs for deleted files and directories. Briefly, the synchronization vectors for a deleted file is the same as the nearest live ancestor and the modification vector need not be stored if we have creation time and the creator for the file stored in the meta-data. This helps us reduce the cost of storing meta-data as shown in the Results section. However, we do not implement the incremental version vectors and synchronization vectors that Tra does, so our meta-data scales worse than Tra.

