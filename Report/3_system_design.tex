\section{System Design}

Tra introduced the novel idea of {\em vector time pairs}, which allows Tra to provide very strong guarantees such as {\em no false positives} and {\em no meta-data for deleted files} among others. Our main system design is completely modelled after Tra.

\subsection{File Watcher}
TnT stores synchronization meta-data in the form of a {\em tree}, which stores modification histories and synchronization histories for every file/directory in the file system. We use the \href{http://godoc.org/code.google.com/p/go.exp/inotify}{\tt inotify} linux package, that allows the system to recognize when a user makes changes to the file system (such as creates/edits/deletes of files/directories) in an interrupt based manner. This allows us to keep the meta-data up-to-date with the latest state of the file system.

\subsection{Synchronization}
Along the lines of Tra, we consider only one-directional synchronization. Any two machines can synchronize their file systems independent of the rest of the machines.

\subsection{Failure Recovery}
We use a two-phase synchronization method to ensure that crashes do not affect the consistency of the system, even if they happen during sync.

\subsection{Meta-data costs}