\section{System Design}

Tra introduced the novel idea of {\em vector time pairs}, which allows Tra to provide very strong guarantees such as {\em no false positives} and {\em no meta-data for deleted files} among others. Our main system design is completely modelled after Tra.  In order to improve upon Tra's robustness and performance, we create additional mechanisms as well.  We improve upon Tra by designing a File Watcher that monitors the user's actions in the background.  In addition, TnT is able to recover from machine crashes without losing updates becuase of it's Failure Recovery mechanism.  We discuss these core aspects of our system below.

\subsection{File Watcher}
TnT structures synchronization meta-data in the form of a {\em tree}, which stores modification histories and synchronization histories for every file/directory in the file system. We use the \href{http://godoc.org/code.google.com/p/go.exp/inotify}{\tt inotify} linux package, that allows the system to recognize when a user makes changes to the file system (such as creates/edits/deletes of files/directories) in an interrupt based manner. This allows us to keep the meta-data up-to-date with the latest state of the file system.

One design challenge associated with this method is that during synchronization, the File Watcher must be able to distinguish between whether the directory is modified by the user or by a synchronization.  Since File Watcher is interrupt based, it must act based on the information in the interrupt.  The {\tt inotify} linux package returns the name of the file and an event mask describing the type of event that occured.  The name and event do not provide enough information.  In order to distinguish between a sync and user action, TnT first copies all of the changes to a tmp folder associated with its directory.  Then, the process copies all of its changes from tmp into the actual directory.  This allows the File Watcher to discriminate between a change made my the user as opposed to the synchronization protocal.  The tmp folder also enables TnT to recover from failures, described in the next 2 sections.

\subsection{Synchronization}
Along the lines of Tra, we consider only one-directional synchronization. Any two machines can synchronize their file systems independent of the rest of the machines.

\subsection{Failure Recovery}
We use a two-phase synchronization method to ensure that crashes do not affect the consistency of the system, even if they happen during sync.

\subsection{Meta-data costs}