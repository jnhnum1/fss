\section{Project Evaluation}

We will test the system developed on up to 5 concurrent machines for:

\begin{itemize}\renewcommand{\labelitemi}{$\diamond$}
\item {\bf Correctness}: The most important aspect of a file synchronizer is that the directories be consistent across machines.  To test the correctness of TnT, we test the case in which each machine independantly updates its respective file directory then randomly synchronizes with another machine.  For this test, the machines are individual processes that watch over seperate file directories on one computer.  The test will manipulate the files and TnT will synchronize it with other machines.  The only way in which two machines can communicate is through the synchronization process.

The test updates a machine's files in a variety of ways on any branch of its directory.  It creates and deletes files and directories, and also modifies the contents of files.  The test does not check for renames or moves because TnT treats each of those actions as a delete then create.  Once the updates and synchronizations are done, the test writes the directories, files and their contents to a string and hashes that string to a number.  If the hashed numbers are the same, then the directories are all up to date.

The test has the machines update their file directories 200 times, then it synchronizes two randomly selected machines.  The process of update then synchronize occurs 100 times totaling 100,000 directory updates per test.  We ran this test 10 times and our results show that the hashes are the same for each of the 10 experiments.


\item {\bf Metadata overhead}: the metadata used at any point of time should be proportional to the size of the file system tree. if files are deleted by all the synchronizing peers, they should not contribute to metadata.

\item {\bf Communications overhead}: The communication used should be proportional to the number of files changed, and the sizes of the files. In particular, it should not be proportional to the size of the file system tree.  
\end{itemize}
